% !TeX root = RJwrapper.tex
\title{Linee guida}
\author{by Ottavia}

\maketitle


\hypertarget{per-il-paper}{%
\section{Per il paper}\label{per-il-paper}}

\begin{itemize}
\item
  ``This includes providing the broader context, implementation details,
  applications or examples of use, with the purpose being to make the
  work relevant to a wider readership than only the package users. The
  paper might explain the novelty in implementation and use of R,
  introduce new data structures or general architectures that could be
  re-usable for other R projects''
\item
  Non più di 20 pagine
\end{itemize}

Nella submission vanno inclusi:

\begin{itemize}
\item
  i file per creare il paper (Rmd, tex, bib, sty le figure e gli output
  (pdf e html))
\item
  I file per riprodurre i risultati presentati nel paper (R scripts e
  dati)
\item
  Una lettera di presentazione Va usato un template specifico che si
  trova nel paccheto \texttt{rjtools} (\texttt{create\_article()})
  \(\rightarrow\) fornisce HTML e PDF
\end{itemize}

Bisogna cercare di usare meno tag possibili nel teso 8e.g., quelle
specifiche solo di html o di PDF)

vogliono i grafici interattivi per la stampa in html


\address{%
Ottavia\\
Unipd\\%
\\
%
%
%
%
}
